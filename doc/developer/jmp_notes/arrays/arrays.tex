\documentclass[a4paper,10pt]{article}
\usepackage{fullpage}
\usepackage{mathtools}
\usepackage{bm}
\usepackage{graphicx}
\usepackage{listings}
\usepackage{hyperref}

\title{Using cctbx dynamic array facilities in c++}
\author{James Parkhurst}

\begin{document}

\maketitle

\newcommand \afversa{\texttt{af::versa}}
\newcommand \afshared{\texttt{af::shared}}
\newcommand \afconstref{\texttt{af::const\_ref}}
\newcommand \afref{\texttt{af::ref}}
\newcommand \afflexgrid{\texttt{af::flex\_grid<>}}
\newcommand \afcgrid{\texttt{af::c\_grid<N>}}

This is a short note on how to use the cctbx dynamic array facilities efficently
from within c++. For a more comprehensive description of the cctbx array 
families see the documentation at \url{http://cci.lbl.gov/~hohn/array-family-tour.htm}.

In cctbx there are a number of ways of using dynamic arrays. You've probably
come across the following classes:

\begin{itemize}
  \item \afshared
  \item \afversa
  \item \afconstref
  \item \afref
\end{itemize}

The \afshared and \afversa interfaces contain a shared pointer
to allow efficient return from functions etc (i.e so they don't have to be
copied, only the pointer gets copied); however, this means that there are
two de-references each time you get an element. The \afconstref and
\afref classes provide direct access to the array data which is 
obviously faster. The \afconstref is for arrays whose elements are
not to be changed. The \afref class is for arrays whose elements can
change.

In addition to these classes, you'll have seen the following classes which
provide multi-dimensional array access:

\begin{itemize}
  \item \afflexgrid
  \item \afcgrid
\end{itemize}

The \afflexgrid class allows multi-dimensional indexing for an
arbitrary number of dimensions (up to 10). The \afcgrid class
has the number of dimensions determined at compile time. These classes
take a multi-dimensional index (i.e. array(i, j)) and convert it to a 1d index.
The \afcgrid class is more efficient if you know the number of
dimensions.

In general, I've found that the following way of using the cctbx array families
seems to work. Where you input an array into a function, use the 
\afconstref and \afref classes. Note that in the following 
example the \afref is passed by value rather than by reference. This
is done because otherwise the flex from python won't work. 

\begin{lstlisting}
void function(const af::const_ref<double> &input, af::ref<double> output) {
  ...
}
\end{lstlisting}

Where you create an array within a function create it as an
\afshared if it has 1 dimension or an \texttt{af::versa<T, af::c\_grid<N> >}
if it has multiple dimensions. You can then take a \afconstref or
\afref within to enable faster access.

\begin{lstlisting}
void function() {
  af::versa<double, c_grid<2> > array(af::c_grid<2>(ysize, xsize));
  af::const_ref<double, c_grid<2> > array_const_ref = array.const_ref();
  af::ref<double, c_grid<2> > array_ref = array.ref();
}
\end{lstlisting}

Where you want to pass an array into a class and then store the array without
copying it, you will have to have a \afversa or \afshared and
in the boost python interface specify a wrapper since flex arrays cannot be
converted to \afversa arrays, only from. For example.

\begin{lstlisting}
class MyClass {
public:
  MyClass(const versa<double, c_grid<2> > &a) : a_(a) {}
private:
  versa<double, c_grid<2> > a_;
};
\end{lstlisting}

For convenience, I've added a header file that imports all the array types:

\begin{lstlisting}
#include <dials/array_family/scitbx_shared_and_versa.h>
\end{lstlisting}

\end{document}
